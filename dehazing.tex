Image dehazing and its similarities to image matting are explored in detail by \cite{he11}. The basic problem is that, in images taken over some distance, floating particles and moisture can lead to dulling of color for more distant objects. This can significantly reduce the quality of the image taken. Dehazing attempts to determine how much this ``haze" has altered the color at each pixel of an image, and to restore the color to what it would have been if unobscured. The general statement of the problem is to find a solution of the form
\[I_i = J_i\textbf{t}_i+A_i(1-\textbf{t}_i)\]
As in image matting, $I$ is the color value at each pixel of the original image, and $t$ is a transparency value at each pixel. In the case of dehazing, $\textbf{t}$ is referred to as the \textbf{medium transmission}, specifying how much of the light from the original source reaches the camera without being scattered by ambient particles. $J$ is the color of the desired image at each pixel, referred to as the \textbf{scene radiance}, while $A$ is global atmospheric light.
\\\\
A prior for this problem is produced by noting the existence of a so-called \textbf{dark channel}. \cite{he11} observes from a set of haze-free daytime images that natural images almost always have at least one color channel (red, green, or blue) close to zero at every pixel. In contrast, the ambient light in images with haze increases every pixel's color value uniformly, preventing the existence of a dark channel where significant haze occurs. By checking how strongly a dark channel exists at each pixel, an initial guess between $0$ and $1$ can be produced at \textit{every} pixel in the image (in contrast to the sparse sketches used in image matting). The image matting problem is then solved using these constraints, and the matting Laplacian from \cite{levin08}. The only difference here is that when Lagrange's method is applied to incorporate the constraints, the $\gamma$ used is taken to be a \textit{small} positive number. This is known as \textbf{soft matting}, and makes the initial constraints weak (necessary, since otherwise we would start with every $\textbf{t}_i$ already specified!).
\\\\
As an interesting side effect, once matting is complete and the medium transmission is known at every pixel, the haze from the image can be used to determine approximately how far from the camera each pixel is. This is because the amount that atmosphere scatters light is fairly constant at any given place. If the rate of scattering is denoted by a \textbf{scattering coefficient} $\beta$, and the depth at each pixel is denoted by $\textbf{d}_i$, then the depth can be found from the equation
\[\textbf{t}_i=e^{-\beta \textbf{d}_k}\]
In practice, the results of this approximation are surprisingly good!